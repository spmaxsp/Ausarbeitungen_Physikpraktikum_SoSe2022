% --------------------- VARIABLEN -------------------------

\newcommand{\COURSE}{Physik und Materialwissenschaften\\ Praktikum Physik \\}
\newcommand{\SEMESTER}{Elektro- und Informationstechnik II}
\newcommand{\STUDENT}{Maximilian Spahn\\ und\\Benjamin Langer}

\newcommand{\HEADDING}{Praktikum Physik}
\newcommand{\SUBHEADDING}{Versuch 3.2: pn-Übergang und Solarzelle}

% ------------------- DEFINITIONEN -----------------------

\documentclass[a4paper]{scrartcl}

\usepackage[utf8]{inputenc}
\usepackage[ngerman]{babel}
\usepackage{amsmath}
\usepackage{amssymb}
\usepackage{color}
\usepackage{tikz}
\usepackage{float}
\usetikzlibrary{arrows,decorations.markings}
\usepackage{tabularx}
\usepackage{fancybox}
\usepackage{pgfplots}
\usepackage{geometry}
\usepackage{fancyhdr}
\usepackage[page]{totalcount}
\usepackage[colorlinks=true,linkcolor=black,urlcolor=blue,bookmarks,bookmarksopen=true]{hyperref}

%Größe der Ränder setzen
\geometry{a4paper,left=2cm, right=2cm, top=3cm, bottom=2cm, headheight=8cm}

%Kopf- und Fußzeile
\pagestyle {fancy}
\fancyhf{}
\fancyhead[L]{\STUDENT}
\fancyhead[C]{\COURSE}
\fancyhead[R]{\today}

\fancyfoot[L]{\SEMESTER}
\fancyfoot[C]{}
\fancyfoot[R]{Seite \thepage /\pageref{LastPage}}

%Formatierung der Überschrift, hier nichts ändern
\def\header#1#2{
  \begin{center}
    {\Large #1}\\
    {#2}
  \end{center}
}

\numberwithin{equation}{subsection}

\setlength\parindent{0pt}

% ----------------------- DOCUMENT ---------------------------

\begin{document}

\vspace{10pt}
\header{\HEADDING}{\SUBHEADDING}

\tableofcontents

\newpage

\section{Einleitung}
In der folgenden Ausarbeitung wird die Funktionsweise einer Solarzelle anhand des Versuches
\glqq pn-Übergang und Solarzelle\grqq \hspace{0cm} messtechnisch untersucht. Der Versuch behandelt
den Unterschied einer unbeleuchteten und einer beleuchteten Solarzelle. Die Ergebnisse werden graphisch
und rechnerisch ausgewertet.
\newpage
\section{Theorie}
%TODO
TODO

\newpage
\section{Häusliche Vorarbeit}
\subsection{An welchen Stellen des I-U Diagramms wird eine höhere Dichte an Messwerten benötigt?}
Am Anfang der Kennlinie von $-2V$ bis $0V$ ist der Verlauf relativ linear. Dadurch werden theoretisch nur
zwei Messwerte und praktisch nur wenige Messwerte benötigt. Das liegt daran, dass die Diode für diesen
Fall in Sperrrichtung geschaltet ist und die Spannung am parallelgeschalteten Widerstand $r_{\text{sh}}$
abfällt (siehe Abbildung \ref{fig:ESB_Solar}). Ab $0V$ wird die Diode nicht mehr in Sperrrichtung sondern
in Flussrichtung geschaltet, wodurch der Anteil des Stromes der durch die Diode fließt zunimmt.
Da die Kennlinie der Diode exponentiell verläuft und der größte Teil der Spannung nun an der Diode
abfällt, muss die Anzahl der Messwerte deutlich erhöht werden.

\begin{figure}[H]
\includegraphics[width=11cm]{ESB_Solarzelle}
\centering
\caption{Ersatzschaltbild der Solarzelle}
\centering
\label{fig:ESB_Solar}
\end{figure}

\subsection{I-U Diagramme für eine unbeleuchtete und beleuchtete Solarzelle}

\begin{figure}[H]
\includegraphics[width=14cm]{Kennlinie}
\centering
\caption{Kennlinien für beleuchtete- und unbeleuchtete Solarzelle}
\centering
\label{fig:Kennlinien}
\end{figure}

\subsection{Einfluss der Widerstände auf den Füllfaktor}
Je größer der Füllfaktor wird, desto mehr ähnelt die Kennlinie einer idealen Stromquelle. Das heißt die
Solarzelle wird effizienter, je näher der Füllfaktor am Wert 1 liegt. Dies wäre der Fall, wenn
der Serienwiderstand gegen 0 $\lim_{r_{\text{s}} \to 0}$ und der Parallelwiderstand gegen unendlich ginge
$\lim_{r_{\text{sh}} \to \infty}$ (siehe Abbildung \ref{fig:ESB_Solar}).

\subsection{Aktueller Stand der Technik bei Solarzellen}
%TODO
TODO

\newpage
\section{Aufbau und Durchführung}
%TODO
TODO

\newpage
\section{Auswertung Versuch}
%TODO
TODO

\newpage
\section{Wertung/Fazit}
%TODO
TODO

\newpage
\section{Anhang}
%TODO
TODO

\newpage
\section{Literatur}
$[$1$]$ 


\label{LastPage}

\end{document}
%%% Local Variables:
%%% mode: latex
%%% TeX-master: t
%%% End:
