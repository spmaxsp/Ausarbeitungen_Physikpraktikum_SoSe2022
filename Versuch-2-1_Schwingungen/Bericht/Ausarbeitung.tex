% --------------------- VARIABLEN -------------------------

\newcommand{\COURSE}{Physik und Materialwissenschaften\\ Praktikum Physik}
\newcommand{\SEMESTER}{Elektro- und Informationstechnik II}
\newcommand{\STUDENT}{Benjamin Langer\\ und\\ Maximilian Spahn}

\newcommand{\HEADDING}{Praktikum Physik}
\newcommand{\SUBHEADDING}{Versuch 2.1: Schwingungen}

% ------------------- DEFINITIONEN -----------------------

\documentclass[a4paper]{scrartcl}

\usepackage[utf8]{inputenc}
\usepackage[ngerman]{babel}
\usepackage{amsmath}
\usepackage{amssymb}
\usepackage{color}
\usepackage{tikz}
\usetikzlibrary{arrows,decorations.markings}
\usepackage{tabularx}
\usepackage{fancybox}
\usepackage{pgfplots}
\usepackage{geometry}
\usepackage{fancyhdr}
\usepackage[page]{totalcount}

%Größe der Ränder setzen
\geometry{a4paper,left=2cm, right=2cm, top=2cm, bottom=2cm}

%Kopf- und Fußzeile
\pagestyle {fancy}
\fancyhead[L]{\STUDENT}
\fancyhead[C]{\COURSE}
\fancyhead[R]{\today}

\fancyfoot[L]{\SEMESTER}
\fancyfoot[C]{}
\fancyfoot[R]{Seite \thepage /\totalpages}

%Formatierung der Überschrift, hier nichts ändern
\def\header#1#2{
  \begin{center}
    {\Large #1}\\
    {#2}
  \end{center}
}


% ----------------------- DOCUMENT ---------------------------

\begin{document}

\header{\HEADDING}{\SUBHEADDING}

\tableofcontents

\section{Häusliche Vorarbeit:}
\subsection{Aufgabe 3.1.1}

\begin{align}
(m*\dfrac{d^2}{dt} + b*\dfrac{dx}{dt} + k*x = 0)
\end{align}

\begin{align*}
 \text{Auslenkung:}& \quad x 
 	&&\Rightarrow \varphi \\
 \text{Masse:}& \quad m 
 	&&\Rightarrow J \textit{(Trägheitsmoment)} &\\
 \text{Geschwindigkeit:}& \quad v=\dfrac{dx}{dt} 
 	&&\Rightarrow \omega \textit{(Winkelgeschwindigeit)}&\\
 \text{Beschleunigung:}& \quad a=\dfrac{d^2x}{dt^2} 
 	&&\Rightarrow \alpha \textit{(Winkelbeschleunigung)}&
\end{align*}

\begin{align*}
 \text{Newton:}& \quad m \cdot a 
 	&&\Rightarrow J \cdot \alpha = J \dfrac{d\varphi}{dt} &\\
 \text{Dämpfungsgrad:}& \quad b \cdot v 
 	&&\Rightarrow b \cdot \omega = J \dfrac{d^2\varphi}{dt^2} &\\
 \text{Beschleunigung:}& \quad k \cdot x 
 	&&\Rightarrow k \cdot \varphi &
\end{align*}

\begin{align}
\Rightarrow \text{DGL.Torsionsschwinger:} \quad J*\dfrac{d^\varphi}{dt} + b*\dfrac{d\varphi}{dt} + k*\varphi = 0
\end{align}

\subsection{Aufgabe 3.1.2}

Definition Drehfederkonstante:
\begin{align}
M = k \cdot \varphi
\end{align}

Definition Drehmoment:
\begin{align}
M = r \cdot F
\end{align}

Federkonstante aus Auslenkung und Kraft am Radius $r$:
\begin{align}
k = \dfrac{r \cdot F}{\varphi}
\end{align}

\begin{table}[]
\begin{tabular}{|l|l|l|}
\hline
\textbf{$\varphi/rad$} & \textbf{$F/N$} & \textbf{$k/\dfrac{Nmm}{rad}$} \\ \hline
0,6                    & 0,1            & 15,83                         \\ \hline
0,8                    & 1,15           & 17,81                         \\ \hline
1,1                    & 0,2            & 17,27                         \\ \hline
1,3                    & 0,25           & 18,26                         \\ \hline
1,6                    & 0,3            & 17,81                         \\ \hline
1,8                    & 0,35           & 18,47                         \\ \hline
2,0                    & 0,4            & 19                            \\ \hline
2,3                    & 0,46           & 19                            \\ \hline
2,4                    & 0,48           & 19                            \\ \hline
\end{tabular}
\end{table}

\begin{align*}
\Rightarrow \text{Federkonstante:} \quad \overline{k} \approx 18,1 \dfrac{Nmm}{rad} = 18,1 \cdot 10^{-3} \dfrac{Nm}{rad}
\end{align*}

\subsection{Aufgabe 3.1.3}

\begin{align}
A_{ges} &= \Pi \cdot r^2 = 28352,87mm^2 \\
A_r &= A_{ges} - A_s = 24872,87mm^2 \\
m_r &= \dfrac{m_{ges}}{A_{ges} \cdot A_r = 22,47g}
\end{align}

Massenträgheitsmoment Hohlzylinder:
\begin{align}
J_r = \dfrac{1}{2} \cdot (r_{\textit{innen}}^2 + r_{\textit{außen}}^2) = 1629,59 kg \cdot mm^2
\end{align}

Massenträgheitsmoment Gesamt:
\begin{align}
J_{ges} = J_{s} + J_{r} = 1829,6 1629,59 kg \cdot mm^2 = 1829,6 \cdot 10^{-6} kg \cdot m^2
\end{align}
	
\subsection{Aufgabe 3.1.3}

Eigenfrequenz:
\begin{align}
\omega_{0,\textit{theor.}} = \sqrt{\dfrac{k}{J}} = 3,145 \dfrac{rad}{s}
\end{align}

Periodendauer:
\begin{align}
T_{0,\textit{theor.}} = \dfrac{2\Pi}{\omega_{0,\textit{theor.}}} = 1,99s
\end{align}

\subsection{Aufgabe 3.2.1}

Eine Spule besteht aus einem (dünnen) gewickelten Draht, welcher selbst einen Leitungswiederstand aufweist.
Dieser kann ersatzweise als Widerstand in Reihe zu der Spule dargestellt werden.

\subsection{Aufgabe 3.2.2}

\begin{align}
R_{ges} = R_1 + R_2 + R_3
\end{align}

\begin{align*}
R_{3,min} = (0+\dots)\Omega &\Rightarrow R_{ges,min} = (6,45+\dots)k\Omega \\
R_{3,max} = (10\mp\dots)k\Omega &\Rightarrow R_{ges,min} = (16,45\mp\dots)k\Omega
\end{align*}

\end{document}
%%% Local Variables:
%%% mode: latex
%%% TeX-master: t
%%% End: